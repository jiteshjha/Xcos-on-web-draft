
%% bare_conf.tex
%% V1.3
%% 2007/01/11
%% by Michael Shell
%% See:
%% http://www.michaelshell.org/
%% for current contact information.
%%
%% This is a skeleton file demonstrating the use of IEEEtran.cls
%% (requires IEEEtran.cls version 1.7 or later) with an IEEE conference paper.
%%
%% Support sites:
%% http://www.michaelshell.org/tex/ieeetran/
%% http://www.ctan.org/tex-archive/macros/latex/contrib/IEEEtran/
%% and
%% http://www.ieee.org/

%%*************************************************************************
%% Legal Notice:
%% This code is offered as-is without any warranty either expressed or
%% implied; without even the implied warranty of MERCHANTABILITY or
%% FITNESS FOR A PARTICULAR PURPOSE!
%% User assumes all risk.
%% In no event shall IEEE or any contributor to this code be liable for
%% any damages or losses, including, but not limited to, incidental,
%% consequential, or any other damages, resulting from the use or misuse
%% of any information contained here.
%%
%% All comments are the opinions of their respective authors and are not
%% necessarily endorsed by the IEEE.
%%
%% This work is distributed under the LaTeX Project Public License (LPPL)
%% ( http://www.latex-project.org/ ) version 1.3, and may be freely used,
%% distributed and modified. A copy of the LPPL, version 1.3, is included
%% in the base LaTeX documentation of all distributions of LaTeX released
%% 2003/12/01 or later.
%% Retain all contribution notices and credits.
%% ** Modified files should be clearly indicated as such, including  **
%% ** renaming them and changing author support contact information. **
%%
%% File list of work: IEEEtran.cls, IEEEtran_HOWTO.pdf, bare_adv.tex,
%%                    bare_conf.tex, bare_jrnl.tex, bare_jrnl_compsoc.tex
%%*************************************************************************

% *** Authors should verify (and, if needed, correct) their LaTeX system  ***
% *** with the testflow diagnostic prior to trusting their LaTeX platform ***
% *** with production work. IEEE's font choices can trigger bugs that do  ***
% *** not appear when using other class files.                            ***
% The testflow support page is at:
% http://www.michaelshell.org/tex/testflow/



% Note that the a4paper option is mainly intended so that authors in
% countries using A4 can easily print to A4 and see how their papers will
% look in print - the typesetting of the document will not typically be
% affected with changes in paper size (but the bottom and side margins will).
% Use the testflow package mentioned above to verify correct handling of
% both paper sizes by the user's LaTeX system.
%
% Also note that the "draftcls" or "draftclsnofoot", not "draft", option
% should be used if it is desired that the figures are to be displayed in
% draft mode.
%
\documentclass[conference]{IEEEtran}
\usepackage{blindtext, graphicx, tabu}
% Add the compsoc option for Computer Society conferences.
%
% If IEEEtran.cls has not been installed into the LaTeX system files,
% manually specify the path to it like:
% \documentclass[conference]{../sty/IEEEtran}





% Some very useful LaTeX packages include:
% (uncomment the ones you want to load)


% *** MISC UTILITY PACKAGES ***
%
%\usepackage{ifpdf}
% Heiko Oberdiek's ifpdf.sty is very useful if you need conditional
% compilation based on whether the output is pdf or dvi.
% usage:
% \ifpdf
%   % pdf code
% \else
%   % dvi code
% \fi
% The latest version of ifpdf.sty can be obtained from:
% http://www.ctan.org/tex-archive/macros/latex/contrib/oberdiek/
% Also, note that IEEEtran.cls V1.7 and later provides a builtin
% \ifCLASSINFOpdf conditional that works the same way.
% When switching from latex to pdflatex and vice-versa, the compiler may
% have to be run twice to clear warning/error messages.






% *** CITATION PACKAGES ***
%
%\usepackage{cite}
% cite.sty was written by Donald Arseneau
% V1.6 and later of IEEEtran pre-defines the format of the cite.sty package
% \cite{} output to follow that of IEEE. Loading the cite package will
% result in citation numbers being automatically sorted and properly
% "compressed/ranged". e.g., [1], [9], [2], [7], [5], [6] without using
% cite.sty will become [1], [2], [5]--[7], [9] using cite.sty. cite.sty's
% \cite will automatically add leading space, if needed. Use cite.sty's
% noadjust option (cite.sty V3.8 and later) if you want to turn this off.
% cite.sty is already installed on most LaTeX systems. Be sure and use
% version 4.0 (2003-05-27) and later if using hyperref.sty. cite.sty does
% not currently provide for hyperlinked citations.
% The latest version can be obtained at:
% http://www.ctan.org/tex-archive/macros/latex/contrib/cite/
% The documentation is contained in the cite.sty file itself.






% *** GRAPHICS RELATED PACKAGES ***
%
\ifCLASSINFOpdf
  % \usepackage[pdftex]{graphicx}
  % declare the path(s) where your graphic files are
  % \graphicspath{{../pdf/}{../jpeg/}}
  % and their extensions so you won't have to specify these with
  % every instance of \includegraphics
  % \DeclareGraphicsExtensions{.pdf,.jpeg,.png}
\else
  % or other class option (dvipsone, dvipdf, if not using dvips). graphicx
  % will default to the driver specified in the system graphics.cfg if no
  % driver is specified.
  % \usepackage[dvips]{graphicx}
  % declare the path(s) where your graphic files are
  % \graphicspath{{../eps/}}
  % and their extensions so you won't have to specify these with
  % every instance of \includegraphics
  % \DeclareGraphicsExtensions{.eps}
\fi
% graphicx was written by David Carlisle and Sebastian Rahtz. It is
% required if you want graphics, photos, etc. graphicx.sty is already
% installed on most LaTeX systems. The latest version and documentation can
% be obtained at:
% http://www.ctan.org/tex-archive/macros/latex/required/graphics/
% Another good source of documentation is "Using Imported Graphics in
% LaTeX2e" by Keith Reckdahl which can be found as epslatex.ps or
% epslatex.pdf at: http://www.ctan.org/tex-archive/info/
%
% latex, and pdflatex in dvi mode, support graphics in encapsulated
% postscript (.eps) format. pdflatex in pdf mode supports graphics
% in .pdf, .jpeg, .png and .mps (metapost) formats. Users should ensure
% that all non-photo figures use a vector format (.eps, .pdf, .mps) and
% not a bitmapped formats (.jpeg, .png). IEEE frowns on bitmapped formats
% which can result in "jaggedy"/blurry rendering of lines and letters as
% well as large increases in file sizes.
%
% You can find documentation about the pdfTeX application at:
% http://www.tug.org/applications/pdftex





% *** MATH PACKAGES ***
%
%\usepackage[cmex10]{amsmath}
% A popular package from the American Mathematical Society that provides
% many useful and powerful commands for dealing with mathematics. If using
% it, be sure to load this package with the cmex10 option to ensure that
% only type 1 fonts will utilized at all point sizes. Without this option,
% it is possible that some math symbols, particularly those within
% footnotes, will be rendered in bitmap form which will result in a
% document that can not be IEEE Xplore compliant!
%
% Also, note that the amsmath package sets \interdisplaylinepenalty to 10000
% thus preventing page breaks from occurring within multiline equations. Use:
%\interdisplaylinepenalty=2500
% after loading amsmath to restore such page breaks as IEEEtran.cls normally
% does. amsmath.sty is already installed on most LaTeX systems. The latest
% version and documentation can be obtained at:
% http://www.ctan.org/tex-archive/macros/latex/required/amslatex/math/





% *** SPECIALIZED LIST PACKAGES ***
%
%\usepackage{algorithmic}
% algorithmic.sty was written by Peter Williams and Rogerio Brito.
% This package provides an algorithmic environment fo describing algorithms.
% You can use the algorithmic environment in-text or within a figure
% environment to provide for a floating algorithm. Do NOT use the algorithm
% floating environment provided by algorithm.sty (by the same authors) or
% algorithm2e.sty (by Christophe Fiorio) as IEEE does not use dedicated
% algorithm float types and packages that provide these will not provide
% correct IEEE style captions. The latest version and documentation of
% algorithmic.sty can be obtained at:
% http://www.ctan.org/tex-archive/macros/latex/contrib/algorithms/
% There is also a support site at:
% http://algorithms.berlios.de/index.html
% Also of interest may be the (relatively newer and more customizable)
% algorithmicx.sty package by Szasz Janos:
% http://www.ctan.org/tex-archive/macros/latex/contrib/algorithmicx/




% *** ALIGNMENT PACKAGES ***
%
%\usepackage{array}
% Frank Mittelbach's and David Carlisle's array.sty patches and improves
% the standard LaTeX2e array and tabular environments to provide better
% appearance and additional user controls. As the default LaTeX2e table
% generation code is lacking to the point of almost being broken with
% respect to the quality of the end results, all users are strongly
% advised to use an enhanced (at the very least that provided by array.sty)
% set of table tools. array.sty is already installed on most systems. The
% latest version and documentation can be obtained at:
% http://www.ctan.org/tex-archive/macros/latex/required/tools/


%\usepackage{mdwmath}
%\usepackage{mdwtab}
% Also highly recommended is Mark Wooding's extremely powerful MDW tools,
% especially mdwmath.sty and mdwtab.sty which are used to format equations
% and tables, respectively. The MDWtools set is already installed on most
% LaTeX systems. The lastest version and documentation is available at:
% http://www.ctan.org/tex-archive/macros/latex/contrib/mdwtools/


% IEEEtran contains the IEEEeqnarray family of commands that can be used to
% generate multiline equations as well as matrices, tables, etc., of high
% quality.


%\usepackage{eqparbox}
% Also of notable interest is Scott Pakin's eqparbox package for creating
% (automatically sized) equal width boxes - aka "natural width parboxes".
% Available at:
% http://www.ctan.org/tex-archive/macros/latex/contrib/eqparbox/





% *** SUBFIGURE PACKAGES ***
%\usepackage[tight,footnotesize]{subfigure}
% subfigure.sty was written by Steven Douglas Cochran. This package makes it
% easy to put subfigures in your figures. e.g., "Figure 1a and 1b". For IEEE
% work, it is a good idea to load it with the tight package option to reduce
% the amount of white space around the subfigures. subfigure.sty is already
% installed on most LaTeX systems. The latest version and documentation can
% be obtained at:
% http://www.ctan.org/tex-archive/obsolete/macros/latex/contrib/subfigure/
% subfigure.sty has been superceeded by subfig.sty.



%\usepackage[caption=false]{caption}
%\usepackage[font=footnotesize]{subfig}
% subfig.sty, also written by Steven Douglas Cochran, is the modern
% replacement for subfigure.sty. However, subfig.sty requires and
% automatically loads Axel Sommerfeldt's caption.sty which will override
% IEEEtran.cls handling of captions and this will result in nonIEEE style
% figure/table captions. To prevent this problem, be sure and preload
% caption.sty with its "caption=false" package option. This is will preserve
% IEEEtran.cls handing of captions. Version 1.3 (2005/06/28) and later
% (recommended due to many improvements over 1.2) of subfig.sty supports
% the caption=false option directly:
%\usepackage[caption=false,font=footnotesize]{subfig}
%
% The latest version and documentation can be obtained at:
% http://www.ctan.org/tex-archive/macros/latex/contrib/subfig/
% The latest version and documentation of caption.sty can be obtained at:
% http://www.ctan.org/tex-archive/macros/latex/contrib/caption/




% *** FLOAT PACKAGES ***
%
%\usepackage{fixltx2e}
% fixltx2e, the successor to the earlier fix2col.sty, was written by
% Frank Mittelbach and David Carlisle. This package corrects a few problems
% in the LaTeX2e kernel, the most notable of which is that in current
% LaTeX2e releases, the ordering of single and double column floats is not
% guaranteed to be preserved. Thus, an unpatched LaTeX2e can allow a
% single column figure to be placed prior to an earlier double column
% figure. The latest version and documentation can be found at:
% http://www.ctan.org/tex-archive/macros/latex/base/



%\usepackage{stfloats}
% stfloats.sty was written by Sigitas Tolusis. This package gives LaTeX2e
% the ability to do double column floats at the bottom of the page as well
% as the top. (e.g., "\begin{figure*}[!b]" is not normally possible in
% LaTeX2e). It also provides a command:
%\fnbelowfloat
% to enable the placement of footnotes below bottom floats (the standard
% LaTeX2e kernel puts them above bottom floats). This is an invasive package
% which rewrites many portions of the LaTeX2e float routines. It may not work
% with other packages that modify the LaTeX2e float routines. The latest
% version and documentation can be obtained at:
% http://www.ctan.org/tex-archive/macros/latex/contrib/sttools/
% Documentation is contained in the stfloats.sty comments as well as in the
% presfull.pdf file. Do not use the stfloats baselinefloat ability as IEEE
% does not allow \baselineskip to stretch. Authors submitting work to the
% IEEE should note that IEEE rarely uses double column equations and
% that authors should try to avoid such use. Do not be tempted to use the
% cuted.sty or midfloat.sty packages (also by Sigitas Tolusis) as IEEE does
% not format its papers in such ways.





% *** PDF, URL AND HYPERLINK PACKAGES ***
%
%\usepackage{url}
% url.sty was written by Donald Arseneau. It provides better support for
% handling and breaking URLs. url.sty is already installed on most LaTeX
% systems. The latest version can be obtained at:
% http://www.ctan.org/tex-archive/macros/latex/contrib/misc/
% Read the url.sty source comments for usage information. Basically,
% \url{my_url_here}.





% *** Do not adjust lengths that control margins, column widths, etc. ***
% *** Do not use packages that alter fonts (such as pslatex).         ***
% There should be no need to do such things with IEEEtran.cls V1.6 and later.
% (Unless specifically asked to do so by the journal or conference you plan
% to submit to, of course. )


% correct bad hyphenation here
\hyphenation{op-tical net-works semi-conduc-tor}


\begin{document}
%
% paper title
% can use linebreaks \\ within to get better formatting as desired
\title{Xcos-on-web: Online design of Scilab-Xcos block schemes}


% author names and affiliations
% use a multiple column layout for up to three different
% affiliations
\author{\IEEEauthorblockN{Michael Shell}
\IEEEauthorblockA{School of Electrical and\\Computer Engineering\\
Georgia Institute of Technology\\
Atlanta, Georgia 30332--0250\\
Email: http://www.michaelshell.org\\\\
\IEEEauthorblockN{Michael Shell}
School of Electrical and\\Computer Engineering\\
Georgia Institute of Technology\\
Atlanta, Georgia 30332--0250\\
Email: http://www.michaelshell.org}
\and
\IEEEauthorblockN{Homer Simpson}
\IEEEauthorblockA{Twentieth Century Fox\\
Springfield, USA\\
Email: homer@thesimpsons.com\\\\
\IEEEauthorblockN{Michael Shell}
School of Electrical and\\Computer Engineering\\
Georgia Institute of Technology\\
Atlanta, Georgia 30332--0250\\}
\and
\IEEEauthorblockN{Jitesh Kumar Jha}
\IEEEauthorblockA{Department of Comptuter Science\\ \& Engineering\\
Manipal Institute of Technology\\
Manipal, Karnataka, India 576104\\
Email: jitesh.kumar2@learner.manipal.edu}}



% conference papers do not typically use \thanks and this command
% is locked out in conference mode. If really needed, such as for
% the acknowledgment of grants, issue a \IEEEoverridecommandlockouts
% after \documentclass

% for over three affiliations, or if they all won't fit within the width
% of the page, use this alternative format:
%
%\author{\IEEEauthorblockN{Michael Shell\IEEEauthorrefmark{1},
%Homer Simpson\IEEEauthorrefmark{2},
%James Kirk\IEEEauthorrefmark{3},
%Montgomery Scott\IEEEauthorrefmark{3} and
%Eldon Tyrell\IEEEauthorrefmark{4}}
%\IEEEauthorblockA{\IEEEauthorrefmark{1}School of Electrical and Computer Engineering\\
%Georgia Institute of Technology,
%Atlanta, Georgia 30332--0250\\ Email: see http://www.michaelshell.org/contact.html}
%\IEEEauthorblockA{\IEEEauthorrefmark{2}Twentieth Century Fox, Springfield, USA\\
%Email: homer@thesimpsons.com}
%\IEEEauthorblockA{\IEEEauthorrefmark{3}Starfleet Academy, San Francisco, California 96678-2391\\
%Telephone: (800) 555--1212, Fax: (888) 555--1212}
%\IEEEauthorblockA{\IEEEauthorrefmark{4}Tyrell Inc., 123 Replicant Street, Los Angeles, California 90210--4321}}




% use for special paper notices
%\IEEEspecialpapernotice{(Invited Paper)}




% make the title area
\maketitle


\begin{abstract}
%\boldmath
The paper presents an online tool that enables design of structures for simulation in virtual laboratories. This is built in the Scilab-Xcos environment. For running such simulations, it is essential to construct a block scheme corresponding to a control of virtual or remote device. The presented tool is a solution which offers this in a comfortable way by allowing users to access experiments over the Internet as it has accomplished porting all the core Xcos functionalities to its browser-only version that can be used without the requirement of installing additional plug-ins or software. It is available to all, 24 hours a day from any place. It is implemented to render an alternate which is an open source equivalent of LabVIEW. It is an interface which can be used to impart practical knowledge through remote experimentation. It’s developed in commonly used web technologies to ensure wide compatibility and platform independence. This application can be used as a supporting tool in online and remote laboratories.
\end{abstract}
% IEEEtran.cls defaults to using nonbold math in the Abstract.
% This preserves the distinction between vectors and scalars. However,
% if the journal you are submitting to favors bold math in the abstract,
% then you can use LaTeX's standard command \boldmath at the very start
% of the abstract to achieve this. Many IEEE journals frown on math
% in the abstract anyway.

% Note that keywords are not normally used for peerreview papers.
\begin{IEEEkeywords}
Scilab, Xcos, Simulation, Online Tool, Virtual Laboratories.
\end{IEEEkeywords}






% For peer review papers, you can put extra information on the cover
% page as needed:
% \ifCLASSOPTIONpeerreview
% \begin{center} \bfseries EDICS Category: 3-BBND \end{center}
% \fi
%
% For peerreview papers, this IEEEtran command inserts a page break and
% creates the second title. It will be ignored for other modes.
\IEEEpeerreviewmaketitle



\section{Introduction}
The significance and need of virtual and online laboratories in the area of technical education is increasing rapidly. Students
and interested users can utilize these laboratories via Web browsers whenever they want to and on any machine that has Internet connection.
LabVIEW is a proprietary development environment by National Instrument(NI). It is an expensive tool with an overhead of AMC. This evolves the need for an alternative which is free and open source. Scilab is an open source software for numerical computation and simulation. It is available for all major operating systems: Linux, Mac OS and Windows.
\subsection{Xcos}
Xcos is an open source graphical simulator/editor available with Scilab to design hybrid dynamical system models. Models can be designed, programmed, loaded, saved, compiled and simulated on Xcos. Xcos is distributed together with SciLab. All Xcos standard blocks are grouped by categories (signal processing, electrical, hydraulics, interpolation, integral, derivative, etc.). The described browser-based application has replica of the block schemes created in SciLab/Xcos desktop application.

\subsection{Xcos on web}
Xcos on web involves porting of all core functionalities in Xcos simulator to a web browser. This application is mainly developed to cater the requirements of users who do not want to undergo tedious installations and want it to be available online on all platforms including mobile web. To have all the features on browser requires mapping the operations from Xcos desktop version. Xcos is primarily programmed using Java and the internal modules written in C. This has to be translated into browser compatible format. All the components: The GUI, the data structures, the styling of XML files, the conversions, the server should be integrated for complete working of the application. All the blocks used in simulation have to be rewritten and restructured to be in accordance with the browser format. Specific but commonly used technologies are chosen in order to keep the widest availability of this tool. There are no special demands from the client users. The computations and simulations occur on the server side. The xcos block diagram is sent as XML file to the server and the result of simulation is received from the server as value or graphical image. The tool also focuses on providing more interactive graphical user interface compared to that of the Xcos desktop version.

\section{Related Work}
This section briefly presents the research literature related to Xcos and xcos related browser based applications.
Sim web-edit is an online tool created to cater the need for virtual simulation applications.[1]

\subsection{Technology Overview}
The tool is based on XHTML, CSS and JavaScript
language. The incorporated export and import is performed
by PHP technology mainly through AJAX requests. The only thing that is required from client side is compatible web browser (the application was tested mainly in FireFox 3.5 and Internet Explorer 8, but it should work with no limitations in IE 6.0+, FF 2+, Safari 3.0+, Opera 9.0+,
Chrome). JavaScript library jQuery is used to speed up the development
and simplify the source code. The library is used for the
object manipulation, changing CSS 2 properties, visual
functions and AJAX requests. The application internally operates with XML and JSON 3 formats. Therefore the support of XML and JSON handling
functions are required. In fact, these functions are natively
supported by PHP version 5.2 or higher, thus it is the only
requirement to the web server. [1]

\subsection{Configuration}
The key part of application is the common configuration file
used by front-end part of the application as well as by the
back-end scripts for export and import. The configuration is
stored in XML format. The XML file contains settings for
each block that is available in the web editor. These settings
include information about attributes that are used for
displaying blocks in the editor: image file name, dimensions,
block class name, input and output ports.
Data contained in the configuration file are used not only for
visualization in web editor’s interface, but also for correct
function of import and export scripts.
The application is easily extendable by editing configuration
file by the application administrator. It is not necessary to edit
source codes of the application. New blocks can be easily
added by modifying configuration XML file. Currently, the
web editor includes the most common blocks used in Xcos
block schemes.[1]

\section{System Architecture}
\begin{figure}
    \centering
    \includegraphics[scale=0.29]{SA}
    \caption{The system architecture- Xcos on web application}
    \label{fig:my_label}
\end{figure}
We have tried to replicate the Xcos Graphical User Interface completely except for their bugs with regard to the wire and port constraints and split connections. Xcos provides users with a palette browser which contains all the standard blocks grouped by categories and an editor where we can drag and drop the required blocks to form the simulation diagram.
Similarly, we provide a sidebar which contains all the Xcos blocks and a
drawing space where users can drag and drop the blocks. For Xcos, Scilab uses JGraphX at the very core and wrap the vertices and edges of the graph with their custom data types and values to make them apt for simulation. They pack all the required data as an XcosDiagram and save it in xcos format.
A xcos file contains all the data which is required
by Scilab engine to simulate the graph.
The data and values for all the blocks are added as soon as the blocks
are being dragged and dropped on the editor. The entire graph object
is then encoded using mxGraph's pre-defined functions. The encoded
XML is very much similar to that of the xcos file but still not
completely the same. The few minor differences are then corrected
using a stylesheet written using XSLT, which contains the rules to be applied whenever a particular tag is encountered. Once the transformation is completed, we have a xcos XML with us. This XML is then sent to the server as a blob object.
The server has a Java servlet which calls the Scilab engine with
proper parameters and the received XML. The simulation is
performed and result is sent back to the server which is then displayed
appropriately on the GUI.



\section{Graphical User Interface}
%Include Screenshot Picture
\begin{figure}
    \centering
    \includegraphics[scale=0.29]{SA}
    \caption{The system architecture- Xcos on web application}
    \label{fig:my_label}
\end{figure}
\iffalse
* --What mxgraph is \\
* --Mxrgaph Js equivalent of JGraphX \\
*
* --Wires connection with mid-point snapping and multiple way-points was implemented on the web. \\
* Created a custom delete function to delete all the edges related to a selected mxCell vertices. \\
* --Implemented connection constraints. (Rules) \\
* Rotate, mirror function \\
* Completed the 'link function' -- Each recently created link in the web application can now be styled depending on the source of each link and a 'name' attribute has been set to uniquely identify each type of link. \\
* edges - text label, label font family, label font size, label text color
* --Split-block is now displayed over it's immediate associated edges. \\
* --Tooltip \\
* --port constraints and rules\\
* --right side frame on editor\\
*zoom, print, redo, undo etx\\
*right click functionality(all)\\
*how the details were mapped with data structures\\
*colour wheel jquery code
*import export button fucntionality not working..that will be explained in server side\\
\fi

The Graphical User Interface(GUI) of xcos-on-web consists of Pallete Sidebar, Graph Container, Toolbar and Bottom bar. Graphical User Interface employs mxGraph - An open source JavaScript diagramming component for realizing different connections between blocks akin to that in Scilab-Xcos. mxGraph is the Javascript variant of Jgraph, which is used for all graph visualization in Scilab-xcos. Scilab-Xcos Pallete Browser(in Scilab) is represented as a sidebar with predefined blocks. Figure 2 shows a basic diagram on the web application.

\subsection{Pallete Sidebar}

Each predefined block consists of a set of explicit input port, explicit output port, implicit input port, implicit output port, control port and command port, data structure and a SVG image. When the Pallete sidebar is created, each block is created in series in accordance to the design of the block. The designs of all the predefined block are contained in a XML file, which is loaded once the Pallete sidebar is created. According to the design of every block, the corresponding ports, data structure and SVG image is created.

\subsection{Graph Container}

The container of xcos-on-web is the area of the front-end where different graphical layouts of blocks and connections could be made. Every entity(Blocks, connections) are represented as Javascript Objects. Any predefined block on the pallete sidebar are draggable onto the container. All blocks on the container are selectable, draggable and re-sizable. Connections can only initiate from a port attached to a block, connections from any point on the block are disabled.

Every block as an associated Tooltip. A tooltip is a pop-up box that appears when the user moves the mouse pointer. The tooltip of a block consists of key-value pair of data derived from the data structure of that block. These key-value pair shows the attributes of the block. These attributes include Block Name, Simulation
UID(Unique ID), different style attributes and types and number of ports.

The container extends beyond the limits of the available screen, enabling expandable diagrams. On the top-right corner of the container, a small window is situated which shows the position of the current view of the container inside a blue frame with respect to the entire container. The blue frame inside this window could be moved around the entire container to change user's current view.

\subsection{Connection}

A connection between two blocks(also called as a 'wire') is an orthogonal link between ports of the two blocks. A connection can be designated as an implicit link, an explicit link or a command-control link depending upon the source port and destination port of the link as shown:

\begin{center}
\begin{tabu} to 0.4\textwidth { | X[c] | X[c] | X[c] | }
 \hline
 Source & Destination & Type of\\ [0.5ex]
 Port & Port & Connection\\ [0.5ex]
 \hline\hline
 Explicit Output Port & Explicit Input Port & Explicit Link \\
 \hline
 Implicit Output Port & Implicit Input Port & Implicit Link \\
 \hline
 Command Port & Control Port & Command-Control Link  \\
 \hline
\end{tabu}
\end{center}

When the user clicks on a port of a block and drags the mouse pointer, a connection or wire is created in the preview stage. As the user drags the mouse pointer, waypoints on the container are created and a wire is constantly created by adding new waypoints to the previous waypoints.
The preview stage ends when the user snaps the mouse pointer onto the destination port with final waypoint(on the destination port) is added to the wire, thus exiting the preview stage.

The style of the connection in Xcos-on-web inherits the "wireEdgeStyle" provided by mxGraph library. "wireEdgeStyle" ensures connections at right angles with each other(orthogonal connections) and enables highlighting of the connection.

\subsection{Port Constraints}
Permissible connections are enforced with port constraints(For instance, a connection with source as Implicit Output Port and the destination as Explicit Input Port is not allowed). Port constraints are implemented by detecting the source and destination of a connection while the connection is in the preview stage(or,before the connection is created). If the connection violates the port constraints, that connection is discarded and the user is cautioned about the violation of port constraints.

\subsection{Split Block}
%Include Split Block Picture
\begin{figure}
    \centering
    \includegraphics[scale=0.29]{SA}
    \caption{The system architecture- Xcos on web application}
    \label{fig:my_label}
\end{figure}
When the source of a new connection(say, Connection 1) is another connection(say, Connection 2), a Split Block is introduced as a source of Connection 1 on the mid-point of Connection 2. A Split Block cuts the connection 2 into two pieces, and serves as a intermediate node for the two pieces of connection 2 and connection 1. A split block is like any other block, but it doesn't serve any computational purpose when the output of a diagram is produced by the Scilab engine. A split block consists of three ports, two implicit/explicit/ output(or command) ports and two implicit/explicit input(or control) port depending on the type of link.

In Xcos-on-web, when the above event occurs, the mid-point of connection 2 is analyzed by retrieving the first waypoint of Connection 2 and a split-block is introduced at the retrieved waypoint. The destination of Connection 2 is set to the input(or command) port of the split-block, depending on the type of Connection 2. A new connection(say, Connection 3) with type as similar to Connection 2 is introduced with source as the output(or control) port of the split-block and the destination as the original destination port of Connection 2. The destination port of Connection 2 is changed to the input(or control) port of the split-block, thereby resulting into two connections. The original waypoints of Connection 2 are divided into two sets, with the dividing point in the original set of waypoints closest to the coordinates of split-block. The first set of waypoints gives the new structure of Connection 2, while the second set of waypoints gives the structure to Connection 3.

For the source port and destination ports of Connection 1, there exists two cases:

\begin{enumerate}
  \item If the source of Connection 1 is a port of a block and the destination of Connection 1 is another connection(Connection 2), then the new destination port for Connection 1 is the output(or command) port of the split-block.

  \item If the source of Connection 1 is another connection(Connection 2) and  and the destination of Connection 1 is a port of a block, then the new source port for Connection 1 is the output(or command) port of the split-block.

\end{enumerate}

\subsection{Delete Function}


\section{Data Structures (include the converter(amits parser), State Diagram)}
The graphical interface can be used to construct model using blocks present in the palette browser. These model can either be formed by requesting the server every time a block is dropped, that is, by using the data structures provided by the Xcos or by making new data structures which is compatible to the Xcos's data structure and derived from it.

Corresponding to each block a sci file is provided by the Xcos which shows the structure of the block. Switch case is used in the blocks where two main cases are the 'define' case where the initial model and the layout of the block are defined, and the 'set' case, where the interfacing function handles the model update. The other cases made for the smooth flow of the data values are 'get' case, where the current layout of the block is displayed, 'internal' case, where an inner block is returned which is used internally of a SuperBlock and 'details' case, which is used for the GUI functions such as tooltip. These sci files are converted to the JavaScript functions along with the various inbuilt funcions of Scilab. New data structures are made to make it compatible with the actual Xcos library.


\subsection{DS DataTypes}
The main data types on which the whole Xcos block are based are mainly ScilabString, ScilabBoolean, ScilabInteger, ScilabDouble and lists(e.g., list, mlist and tlist). Every Scilab data type consists of at least 3 fields that are height and width of the data type and their corresponding data.
These data types can be considered as the building blocks of the Xcos. Lists are mere array of objects whose structure are modified by XSLT so that the XML file so generated can be used as XCOS file needed to execute the diagram formed at the backend. The objects can be any Scilab data type or another list itself.

\subsection{Functions}
The data type modified at the client end required some functions apart from the operations that are provided by the Scilab because the ways the data type are implemented are quite different. Hence the assistance needed for the easy flow and efficient/optimized coding structure are provided by some new functions like getData, isEmpty etc. The basic aim of these functions was to fetch or perform some operation on the 'data' part of the Scilab Data Types.

\subsection{Operators}
The implicit Scilab functions which are used in the Xcos blocks are re-implemented to suite this new data structure. Various operators/functions are sci2exp, size, ones, zeroes, colon operator etc. Aldready implemented Javascript math libraries are used to make the process simpler and efficient. Using the Scilab operators directly would have been a tedious task because that would have involved requesting the server to evaluate it each a such operator is scanned or used. Hence the way out is to implement such operators on the client side to reduce the time complexity.


\section{XSLT}
\blindtext

\subsection{Subsection Heading Here}
\blindtext

\section{Server Components (include java servlet, AJAX code on GUI side), xcos API( instance, executing), export, import}
\blindtext

\subsection{Subsection Heading Here}
\blindtext

% needed in second column of first page if using \IEEEpubid
%\IEEEpubidadjcol

% An example of a floating figure using the graphicx package.
% Note that \label must occur AFTER (or within) \caption.
% For figures, \caption should occur after the \includegraphics.
% Note that IEEEtran v1.7 and later has special internal code that
% is designed to preserve the operation of \label within \caption
% even when the captionsoff option is in effect. However, because
% of issues like this, it may be the safest practice to put all your
% \label just after \caption rather than within \caption{}.
%
% Reminder: the "draftcls" or "draftclsnofoot", not "draft", class
% option should be used if it is desired that the figures are to be
% displayed while in draft mode.
%
%\begin{figure}[!t]
%\centering
%\includegraphics[width=2.5in]{myfigure}
% where an .eps filename suffix will be assumed under latex,
% and a .pdf suffix will be assumed for pdflatex; or what has been declared
% via \DeclareGraphicsExtensions.
%\caption{Simulation Results}
%\label{fig_sim}
%\end{figure}

% Note that IEEE typically puts floats only at the top, even when this
% results in a large percentage of a column being occupied by floats.


% An example of a double column floating figure using two subfigures.
% (The subfig.sty package must be loaded for this to work.)
% The subfigure \label commands are set within each subfloat command, the
% \label for the overall figure must come after \caption.
% \hfil must be used as a separator to get equal spacing.
% The subfigure.sty package works much the same way, except \subfigure is
% used instead of \subfloat.
%
%\begin{figure*}[!t]
%\centerline{\subfloat[Case I]\includegraphics[width=2.5in]{subfigcase1}%
%\label{fig_first_case}}
%\hfil
%\subfloat[Case II]{\includegraphics[width=2.5in]{subfigcase2}%
%\label{fig_second_case}}}
%\caption{Simulation results}
%\label{fig_sim}
%\end{figure*}
%
% Note that often IEEE papers with subfigures do not employ subfigure
% captions (using the optional argument to \subfloat), but instead will
% reference/describe all of them (a), (b), etc., within the main caption.


% An example of a floating table. Note that, for IEEE style tables, the
% \caption command should come BEFORE the table. Table text will default to
% \footnotesize as IEEE normally uses this smaller font for tables.
% The \label must come after \caption as always.
%
%\begin{table}[!t]
%% increase table row spacing, adjust to taste
%\renewcommand{\arraystretch}{1.3}
% if using array.sty, it might be a good idea to tweak the value of
% \extrarowheight as needed to properly center the text within the cells
%\caption{An Example of a Table}
%\label{table_example}
%\centering
%% Some packages, such as MDW tools, offer better commands for making tables
%% than the plain LaTeX2e tabular which is used here.
%\begin{tabular}{|c||c|}
%\hline
%One & Two\\
%\hline
%Three & Four\\
%\hline
%\end{tabular}
%\end{table}


% Note that IEEE does not put floats in the very first column - or typically
% anywhere on the first page for that matter. Also, in-text middle ("here")
% positioning is not used. Most IEEE journals use top floats exclusively.
% Note that, LaTeX2e, unlike IEEE journals, places footnotes above bottom
% floats. This can be corrected via the \fnbelowfloat command of the
% stfloats package.



\section{Conclusion}
\blindtext

\section{Future Work}
\blindtext




% if have a single appendix:
%\appendix[Proof of the Zonklar Equations]
% or
%\appendix  % for no appendix heading
% do not use \section anymore after \appendix, only \section*
% is possibly needed

% use appendices with more than one appendix
% then use \section to start each appendix
% you must declare a \section before using any
% \subsection or using \label (\appendices by itself
% starts a section numbered zero.)
%


\appendices
\section{Proof of the First Zonklar Equation}
\blindtext

% use section* for acknowledgement
\section*{Acknowledgment}


The authors would like to thank FOSSEE


% Can use something like this to put references on a page
% by themselves when using endfloat and the captionsoff option.
\ifCLASSOPTIONcaptionsoff
  \newpage
\fi



% trigger a \newpage just before the given reference
% number - used to balance the columns on the last page
% adjust value as needed - may need to be readjusted if
% the document is modified later
% The "triggered" command can be changed if desired:
%\IEEEtriggercmd{\enlargethispage{-5in}}

% references section

% can use a bibliography generated by BibTeX as a .bbl file
% BibTeX documentation can be easily obtained at:
% http://www.ctan.org/tex-archive/biblio/bibtex/contrib/doc/
% The IEEEtran BibTeX style support page is at:
% http://www.michaelshell.org/tex/ieeetran/bibtex/
%\bibliographystyle{IEEEtran}
% argument is your BibTeX string definitions and bibliography database(s)
%\bibliography{IEEEabrv,../bib/paper}
%
% <OR> manually copy in the resultant .bbl file
% set second argument of \begin to the number of references
% (used to reserve space for the reference number labels box)
\begin{thebibliography}{1}

\bibitem{IEEEhowto:kopka}
H.~Kopka and P.~W. Daly, \emph{A Guide to \LaTeX}, 3rd~ed.\hskip 1em plus
  0.5em minus 0.4em\relax Harlow, England: Addison-Wesley, 1999.

\end{thebibliography}

% biography section
%
% If you have an EPS/PDF photo (graphicx package needed) extra braces are
% needed around the contents of the optional argument to biography to prevent
% the LaTeX parser from getting confused when it sees the complicated
% \includegraphics command within an optional argument. (You could create
% your own custom macro containing the \includegraphics command to make things
% simpler here.)
%\begin{biography}[{\includegraphics[width=1in,height=1.25in,clip,keepaspectratio]{mshell}}]{Michael Shell}
% or if you just want to reserve a space for a photo:

\begin{IEEEbiography}[{\includegraphics[width=1in,height=1.25in,clip,keepaspectratio]{picture}}]{John Doe}
\blindtext
\end{IEEEbiography}

% You can push biographies down or up by placing
% a \vfill before or after them. The appropriate
% use of \vfill depends on what kind of text is
% on the last page and whether or not the columns
% are being equalized.

%\vfill

% Can be used to pull up biographies so that the bottom of the last one
% is flush with the other column.
%\enlargethispage{-5in}




% that's all folks
\end{document}
